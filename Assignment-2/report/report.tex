\documentclass[12pt]{article}

\usepackage[letterpaper, hmargin=0.75in, vmargin=0.75in]{geometry}
\usepackage{float}
\usepackage{listings}
\usepackage{xcolor}
 
\definecolor{codegreen}{rgb}{0,0.6,0}
\definecolor{codegray}{rgb}{0.5,0.5,0.5}
\definecolor{codepurple}{rgb}{0.58,0,0.82}
\definecolor{backcolour}{rgb}{0.95,0.95,0.92}
 
\lstdefinestyle{mystyle}{
    backgroundcolor=\color{backcolour},   
    commentstyle=\color{codegreen},
    keywordstyle=\color{magenta},
    numberstyle=\tiny\color{codegray},
    stringstyle=\color{codepurple},
    basicstyle=\ttfamily\footnotesize,
    breakatwhitespace=false,         
    breaklines=true,                 
    captionpos=b,                    
    keepspaces=true,                 
    numbers=left,                    
    numbersep=5pt,                  
    showspaces=false,                
    showstringspaces=false,
    showtabs=false,                  
    tabsize=2
}
 
\lstset{style=mystyle}

\pagestyle{empty}

\title{ECE 459: Programming for Performance\\Assignment 2}
\author{Your Name}
\date{\today}

\begin{document}

\maketitle

{\bf I verify I ran all benchmarks on a Intel(R) Xeon(R) Gold 5120 CPU with 14 physical cores and
{\tt OMP\_NUM\_THREADS} set to 14 (I double checked with
{\tt echo \$OMP\_NUM\_THREADS})}

\section*{Automatic Parallelization (15 marks)}

\begin{table}[H]
  \centering
  \begin{tabular}{lr}
    & {\bf Time (s)} \\
    \hline
    Run 1 & 8.461 \\
    Run 2 & 8.223 \\
    Run 3 & 8.898 \\
    \hline
    Average & 8.527 \\
  \end{tabular}
  \caption{Benchmark results for raytrace unoptimized sequential execution}
  \label{tbl-raytrace-unopt-sequential}
\end{table}

\begin{table}[H]
  \centering
  \begin{tabular}{lr}
    & {\bf Time (s)} \\
    \hline
    Run 1 & 4.586 \\
    Run 2 & 4.543 \\
    Run 3 & 4.515 \\
    \hline
    Average & 4.548 \\
  \end{tabular}
  \caption{Benchmark results for raytrace optimized sequential execution}
  \label{tbl-raytrace-opt-sequential}
\end{table}

\begin{table}[H]
  \centering
  \begin{tabular}{lr}
    & {\bf Time (s)} \\
    \hline
    Run 1 & 0.5839 \\
    Run 2 & 0.6124 \\
    Run 3 & 0.5846 \\
    \hline
    Average & 0.5936 \\
  \end{tabular}
  \caption{Benchmark results for raytrace with automatic parallelization}
  \label{tbl-raytrace-automatic}
\end{table}

\clearpage
 \begin{enumerate}
    \item My change:
    
\begin{lstlisting}[language=C]
/* Subtract two vectors and return the resulting vector */
#define vectorSub(v1, v2) (vector) {*v1.x - *v2.x, *v1.y - *v2.y, *v1.z - *v2.z }

/* Multiply two vectors and return the resulting scalar (dot product) */
#define vectorDot(v1, v2) (float)(*v1.x * *v2.x + *v1.y * *v2.y + *v1.z * *v2.z)

/* Calculate Vector x Scalar and return resulting Vector*/ 
#define vectorScale(c, v) (vector) {*v.x * c, *v.y * c, *v.z * c }

/* Add two vectors and return the resulting vector */
#define vectorAdd(v1, v2) (vector) {*v1.x + *v2.x, *v1.y + *v2.y, *v1.z + *v2.z }
\end{lstlisting}
I have converted all the vector operation functions into equivalent macros. 

   \item Justify my change
       \begin{enumerate}
         \item Why the existing code does not parallelize?
           \begin{itemize}
             \item In the existing code, there are function calls in the double for loop. The problem with calling another function is that the compiler has no idea what that routine might do. Therefore, a loop that contains function calls cannot, in general, be automatically parallelized.
           \end{itemize}
         \item Why my changes improve parallelization and preserve the behaviour of the sequential version?
           \begin{itemize}
             \item After vector-operation functions are converted to macros, those functions are handled by the pre-compiler and guaranteed to be inlined. By inlining functions using Macros, the double for-loop containing those function calls are parallelized automatically.
           \end{itemize}
           \begin{itemize}
             \item My change preserves the behaviour of the sequential version. This is because when changing the vector operation functions to macros and changing the parameter from pointer to value, it preserves the function behaviour.
           \end{itemize}
         \item Why my changes adversely impact maintainability?
           \begin{itemize}
             \item Changing a function to a marco is not maintainable. The first reason is that macro does not support namespaces, which means those Macro names will collide with the other variable or function names and cause unwanted behaviour. The second reason is that Macros can't be debugged. A "function-like" macros will act like a single statement, which makes it hard to figure out what is going on. The third reason is that the address cannot be passed as a parameter to a "function-like" macros.
           \end{itemize}
       \end{enumerate}
   \item Speed up Explanation
   \begin{itemize}
     \item There are around 14x speedup over bin/raytrace, and 7x speedup over bin/raytrace\_opt. This result is as expected. The double for loop was auto parallelized. The work of the for loop is evenly distributed to 14 threads since the OMP\_NUM\_THREADS environment variable is 14. This confirms the 14x speedup over bin/raytrace. Since bin/raytrace\_opt has 2x faster than bin/raytrace, bin/raytrace\_auto has 14 / 2 = 7x speedup over bin/raytrace\_opt.
   \end{itemize}
 \end{enumerate}

%% Refer to {\bf Table~\ref{tbl-raytrace-unopt-sequential}}, {\bf
%% Table~\ref{tbl-raytrace-opt-sequential}}, and {\bf
%% Table~\ref{tbl-raytrace-automatic}} in your explanation.

\section*{Using OpenMP Tasks (30 marks)}

\begin{table}[H]
  \centering
  \begin{tabular}{lr}
    & {\bf Time (s)} \\
    \hline
    Run 1 & 1.792 \\
    Run 2 & 1.897 \\
    Run 3 & 1.823 \\
    Run 4 & 1.733 \\
    Run 5 & 1.742 \\
    Run 6 & 1.758 \\
    \hline
    Average & 1.791 \\
  \end{tabular}
  \caption{Benchmark results for n-queens execution ({\tt n} = \fibNumber{13})}
  \label{tbl-fib-sequential}
\end{table}

\begin{table}[H]
  \centering
  \begin{tabular}{lr}
    & {\bf Time (s)} \\
    \hline
    Run 1 & 0.3367 \\
    Run 2 & 0.3434 \\
    Run 3 & 0.3426 \\
    Run 4 & 0.3440 \\
    Run 5 & 0.3323 \\
    Run 6 & 0.3394 \\
    \hline
    Average & 0.3397 \\
  \end{tabular}
  \caption{Benchmark results for n-queens execution with OpenMP tasks ({\tt n} = \fibNumber{13})}
  \label{tbl-fib-tasks}
\end{table}

\begin{table}[H]
  \centering
  \begin{tabular}{lr}
    & {\bf Time (s)} \\
    \hline
    Run 1 & 11.318 \\
    Run 2 & 11.815 \\
    Run 3 & 11.259 \\
    Run 4 & 11.687 \\
    Run 5 & 11.565 \\
    Run 6 & 11.453 \\
    \hline
    Average & 11.516 \\
  \end{tabular}
  \caption{Benchmark results for n-queens execution ({\tt n} = \fibNumber{14})}
  \label{tbl-fib-sequential}
\end{table}

\begin{table}[H]
  \centering
  \begin{tabular}{lr}
    & {\bf Time (s)} \\
    \hline
    Run 1 & 1.749 \\
    Run 2 & 1.729 \\
    Run 3 & 1.721 \\
    Run 4 & 1.736 \\
    Run 5 & 1.744 \\
    Run 6 & 1.731 \\
    \hline
    Average & 1.735 \\
  \end{tabular}
  \caption{Benchmark results for n-queens execution with OpenMP tasks ({\tt n} = \fibNumber{14})}
  \label{tbl-fib-tasks}
\end{table}


\clearpage
 \begin{enumerate}
    \item My change:
    
\begin{lstlisting}[language=C]
#pragma omp parallel
{
    #pragma omp single
    {
        for (int j=0; j<n; j++) 
        {
            #pragma omp task firstprivate(j)
            {
                char *new_config;
                new_config = malloc((1)*sizeof(char));
                new_config[0] = j;

                nqueens(new_config, n, 1);

                free(new_config);
            }
        }
    }
}
\end{lstlisting}
To parallelize the solving of an n-queens problem, I have divided the n-queens problem into n sub-problems, where n is the possible positions for the first row of the queen matrix. More specifically, each sub-problem is a new n-queens problem which has the first row filled out. Each sub-problems is an OpenMP task, which can be parallelized.

   \item Justify my change
       \begin{enumerate}
         \item Analyzing the performance of the provided version
           \begin{itemize}
             \item The provided solution is a sequential version that brutal forced the n-queens problem. The provided sequential version has an O($n^n$) time complexity, which is an NP problem. In the recursion tree, the first level has n nodes, because there are n possible positions on the first row. For each possible position on the first row, there are n possible positions on the second row. Therefore, there are $n^2$ nodes on the second level. Since there are total n rows in the queen matrix, there are n levels. Therefore, the time complexity of the provided version is $n^n$, and this explains why the performance is so bad.
           \end{itemize}
         \item Why my changes improved performance?
           \begin{itemize}
             \item After dividing the n-queens problem into n sub-problems, the n sub-problems are evenly distributed to 14 threads. And each sub-problem has only n - 1 levels and time complexity of O($n ^ {n - 1}$). Therefore, the total time complexity is optimized to O($(n / 14) * (n ^ {n - 1})$) = O($(n ^ n) / 14$)
           \end{itemize}
         \item How you could further improve performance?
           \begin{itemize}
             \item My solution can be further improved by creating new tasks from the running thread and keeping the early exited thread working. More specifically, some sub-problems exit earlier and the corresponding thread becomes idle. Therefore, we can create a new task whenever the omp\_set\_num\_threads() is less than 14. By doing this, we can keep all 14 threads working consistently and better distribute the workload.
           \end{itemize}
       \end{enumerate}
       \clearpage
   \item Speed up explanation
   \begin{itemize}
     \item For both n = 13 and 14, there is around 7x speedup over the provided sequential solution. This is as expected. When n = 14, the provided solution has O($14^{14}$) time complexity, and the parallelized solution has O($14^{13}$) time complexity. The theoretical speedup is 14. However, some tasks are finished early, and the corresponding threads become idle. Even though the sub-problems are distributed to 14 threads, some sub-problems have higher time complexity on the constant level, which means the total workload is not evenly distributed to 14 threads. Therefore, the actual speedup is 7x, which is less than the theoretical speedup.
   \end{itemize}
 \end{enumerate}


%% Refer to {\bf Table~\ref{tbl-fib-sequential}} and
%% {\bf Table~\ref{tbl-fib-tasks}} in your explanation.

\section*{Manual Parallelization with OpenMP (55 marks)}

I placed the following OpenMP directives in my program:\\
\\
The following directives were effective:
\begin{itemize}
\item 
\begin{lstlisting}[language=C]
#pragma omp parallel for
\end{lstlisting}
This parallel-for directive works. Inside the first recursion call, there is a for-loop that loop through all the character in the gAlphabet. For each iteration, a new secret is generated and checked if it is a valid secret, and the function recursively calls itself with the new secret. By using the parallel-for directive in front of the for-loop, we can distribute the 36 units of work to 14 threads for first recursion call. And later on, when one thread becomes idle, the parallel-for directive will detect the idle thread and distribute the for-loop workload again to the idle thread, which guarantees that all 14 threads are operating till the end.
\item 
\begin{lstlisting}[language=C]
#pragma omp parallel
{
    #pragma omp single
    {
        for (int i = 0; i < gAlphabet.size(); i++)
        {
            #pragma omp task
            {
                std::string new_secret(1, gAlphabet[i]);

                if (isValidSecret(message, origSig, new_secret)) {
                    cout << new_secret << endl;
                    found = 1;
                }

                dfs(gAlphabet, new_secret, gMaxSecretLen);
            }
        }
    }
}
\end{lstlisting}
Using OpenMP tasks works. Similar to Part 2, to parallelize the generation of all possible secret, I have divided the problem into n sub-problems, where n is the number of possible characters in the JWT’s secret. In my testing, n is 36. More specifically, each sub-problem is a new secret generation problem that has the first character filled out. Each sub-problems is an OpenMP task, which can be parallelized. I use one thread to create all 36 tasks. And all 36 tasks are queued and will be handled concurrently by 14 threads. Therefore, parallelization is achieved using explicit OpenMP tasks.
\end{itemize}\\
\\
I tried these directives and they were not effective:
\begin{itemize}
\item 
\begin{lstlisting}[language=C]
#pragma omp for ordered
\end{lstlisting}
The Ordered directive does not work. Without the parallel directive, the ordered directive by itself does not have the parallelization effect. In our case, inside the for-loop that loop through all the possible characters, the new secret verification is independent of each other. Therefore, there is no point in using the ordered directive.
\item 
\begin{lstlisting}[language=C]
#pragma omp single
\end{lstlisting}
The Single directive does not work. Without the parallel directive, the Single directive by itself does not have the parallelization effect. When using the single directive, only a single thread executes the bottleneck for-loop region. Therefore, there won't be any parallelization for using Single directive.
\item
\begin{lstlisting}[language=C]
#pragma omp master
\end{lstlisting}
The Master directive does not work. Similar to Single directive, without the parallel directive, the Master directive by itself does not have the parallelization effect. When using the master direction, only the master thread executes the bottleneck for-loop region, which is the same as the sequential version. Therefore, using Master directive by itself does not work.
\item
\begin{lstlisting}[language=C]
#pragma omp parallel sections
\end{lstlisting}
The Sections directive does not work, even though the section directives inside the sections directive can be parallelized. This is because the task we want to parallel is a for-loop and create multiple section directions using for loop throws an error. One way to overcome this is to flatten out the for-loop, but this is impractical in programming since the number of possible characters can vary.
\end{itemize}\\
\\
With the parallel-for directive applied, here are the results:

\begin{table}[H]
  \centering
  \begin{tabular}{lr}
    & {\bf Time (s)} \\
    \hline
    Run 1 & 0.183 \\
    Run 2 & 0.188 \\
    Run 3 & 0.178 \\
    Run 4 & 0.191 \\
    Run 5 & 0.185 \\
    Run 6 & 0.184 \\
    \hline
    Average & 0.185 \\
  \end{tabular}
  \caption{Benchmark results for jwtcracker\_sin (sequential version) ({\tt length of secret} = \zetaIterations{4})}
  \label{tbl-zeta-openmp}
\end{table}

\begin{table}[H]
  \centering
  \begin{tabular}{lr}
    & {\bf Time (s)} \\
    \hline
    Run 1 & 2.579 \\
    Run 2 & 2.584 \\
    Run 3 & 2.604 \\
    Run 4 & 2.593 \\
    Run 5 & 2.605 \\
    Run 6 & 2.591 \\
    \hline
    Average & 2.593 \\
  \end{tabular}
  \caption{Benchmark results for jwtcracker\_omp with manual OpenMP ({\tt length of secret} = \zetaIterations{4})}
  \label{tbl-zeta-openmp}
\end{table}

\begin{table}[H]
  \centering
  \begin{tabular}{lr}
    & {\bf Time (s)} \\
    \hline
    Run 1 & 7.375 \\
    Run 2 & 7.290 \\
    Run 3 & 7.790 \\
    Run 4 & 7.072 \\
    Run 5 & 7.858 \\
    Run 6 & 7.281 \\
    \hline
    Average & 7.444 \\
  \end{tabular}
  \caption{Benchmark results for jwtcracker\_sin (sequential version) ({\tt length of secret} = \zetaIterations{5})}
  \label{tbl-zeta-openmp}
\end{table}

\begin{table}[H]
  \centering
  \begin{tabular}{lr}
    & {\bf Time (s)} \\
    \hline
    Run 1 & 102.424 \\
    Run 2 & 102.183 \\
    Run 3 & 101.134 \\
    Run 4 & 101.783 \\
    Run 5 & 102.165 \\
    Run 6 & 102.672 \\
    \hline
    Average & 102.060 \\
  \end{tabular}
  \caption{Benchmark results for jwtcracker\_omp with manual OpenMP ({\tt length of secret} = \zetaIterations{5})}
  \label{tbl-zeta-openmp}
\end{table}

\begin{table}[H]
  \centering
  \begin{tabular}{lr}
    & {\bf Time (s)} \\
    \hline
    Run 1 & 313 \\
    Run 2 & 301 \\
    Run 3 & 305 \\
    \hline
    Average & 306 \\
  \end{tabular}
  \caption{Benchmark results for jwtcracker\_sin (sequential version) ({\tt length of secret} = \zetaIterations{6})}
  \label{tbl-zeta-openmp}
\end{table}

\begin{table}[H]
  \centering
  \begin{tabular}{lr}
    & {\bf Time (s)} \\
    \hline
    Run 1 & 4041 \\
    Run 2 & 4118 \\
    Run 3 & 4154 \\
    \hline
    Average & 4104 \\
  \end{tabular}
  \caption{Benchmark results for jwtcracker\_omp with manual OpenMP ({\tt length of secret} = \zetaIterations{6})}
  \label{tbl-zeta-openmp}
\end{table}\\
\\
Speed up explanation:
   \begin{itemize}
     \item For all secret length of 4, 5 and 6, the jwtcracker\_omp has around 14x speedup over the single-thread solution. This is as expected. For the first recursion call, all 36 units of work are distributed to 14 threads. And later on, when one thread becomes idle, the parallel-for directive will detect the idle thread and distribute the workload to the idle thread, which guarantees that all 14 threads are operating till the end. This confirms that there is a 14x speedup over the provided single-thread solution.
   \end{itemize}
\end{document}
